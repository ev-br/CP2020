\documentclass[10pt]{article}
\usepackage{amssymb}
\usepackage{graphicx,amsmath}
  \usepackage[utf8]{inputenc}
  \usepackage[russian]{babel}
\usepackage{hyperref}
\usepackage{listings}

\textwidth 16.5cm \oddsidemargin 0.5cm \textheight 21cm \topmargin
1cm
\newcommand{\norm}[1]{\left\lVert#1\right\rVert}
\begin{document}

\begin{enumerate}
\item \textbf{(7)} Докажите следующие неравенства и приведите примеры вектора $x$ и матрицы $A$, при которых эти неравенства достигаются:
\begin{itemize}
\item $\norm{x}_2 \le \sqrt{m}\norm{x}_\infty$
\item $\norm{A}_\infty \le \sqrt{n} \norm{A}_2$
\end{itemize}
где $x$ -- вектор длины $m$ и $A$ -- матрица размера $m\times n$.

\item \textbf{(10)} Постройте руками SVD разложение следующих матриц:
$$
(a)\quad\begin{bmatrix}
3 & 0\\
0 & -2
\end{bmatrix},\quad
(b)\quad\begin{bmatrix}
0 & 2\\
0 & 0\\
0 & 0
\end{bmatrix},\quad
(c)\quad\begin{bmatrix}
1 & 1\\
1 & 1
\end{bmatrix}.
$$

\item \textbf{(15)} Напишите код на Python, который для данной действительной матрицы $A$ строит правые сингулярные вектора $v_1,\; v_2$ (вписанные в окружность) и  левые сингулярные вектора $u_1,\; u_2$ (вписанные в эллипс) -- аналогично Fig. 4.1 Trefethen, Bau. Используйте этот код для матриц из Задачи 2.

\item \textbf{(15)} Рассмотрите матрицы $X$ размером $N\times m$, $ \Omega$ размером $m\times m$ и $ \Delta$ размером $n\times n$. Пусть $V = X \Omega X^T + \Delta$  и
$$
f(A) = A^{-1}X(X^T A^{-1}X)^{-1}.
$$
Докажите, предполагая обратимость участвующих матриц, что $f(V)=f(\Delta)$.

\item \textbf{(20)} \href{https://github.com/ev-br/CP2020/blob/master/week_1_LU_pivoting.ipynb}{Реализуйте LU разложение квадратной матрицы с выбором главного элемента}, следуя инструкциям по ссылке.

\item \textbf{(20)} Ознакомьтесь с \href{https://en.wikipedia.org/wiki/Woodbury_matrix_identity}{Woodbury matrix identity}, справедливом для матриц подходящих размеров:
\begin{equation}
\label{wb}
\left(A+UCV\right)^{-1}=A^{-1}-A^{-1}U\left(C^{-1}+VA^{-1}U\right)^{-1}VA^{-1}.
\end{equation}
Рассмотрите частный случай диагональной ($p\times p$) матрицы $A$ и единичной ($k\times k$) матрицы $C$ и напишите функцию `woodbury(A, U, V)' вычисляющую $\left(A+UV\right)^{-1}$ по формуле (\ref{wb}). Проверьте, что Ваша имплементация верна, сравнивая результат с полученным прямолинейным вычислением.
Сравните быстродействие этих двух способов: какой оказывается быстрее и почему? (рассмотрите случайные матрицы с $p = 5000, k = 100$).




\end{enumerate}

\end{document}